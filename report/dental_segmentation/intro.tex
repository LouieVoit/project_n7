\subsubsection*{Abstract}
Dental image processing, like teeth extraction from dental radiography, could have many applications in the forensic field for dental recognition or in dental diagnostic. The main problem is that there is yet not any automatic tool extracting tooth from a dental radiography, so our goal here was to find a method to extraxt those teeth.\\
Since radiography are black and white picture, a pixel is represented in the following by its coordinates in the picture and by its gray intensity. Segmentation method can be classify in two categories : in the first category, methods are using region's statistics to cluster parts of the radiography, in the second category, methods are using edge detection.\\  
Our first research were about the active contour method (method of the first categorie) since traditional edge detection method weren't effective at all on picture like radiography. We were to found a stopping criterion and had to improve the method to make it work on radiography. What we found out is that, since radiography are often really different, the active contour method could work perfectly on one radiography with certain parameters but could not work at all on another radiography with the exact same parameters. That's why, in section 3, we tried to create pre-treatment method aiming at making the active contour method works on every radiography with the same parameters, to make an automatic method.\\
Finally, we tried to gather all of our method to create an automatic tool in section 4. 
\subsubsection*{Acknowloedgments}
As our first research internship, we would like to thank M.Balsa for his guidance and kindness, not only during the work, but also outside of the university. Thanks to his advices and his company, we really enjoyed our time in Bragance.   