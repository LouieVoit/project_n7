Le sujet du stage se situe dans le contexte des accidents graves dans les réacteurs nucléaires. Le stage se déroule au CEA de Cadarache dans le département de Technologie Nucléaire (DTN), dans le laboratoire de Physique et de Modélisation des Accidents graves (LPMA) sous la direction de Laurent Saas. L'étude des accidents graves a débuté trés récemment et le nombre de personne travaillant est encore à ce jour trés faible. De plus, les connaissances physiques au sujet des accidents graves sont encore très limitées et du fait du risque que présentent les réactions nucléaires, la collecte de nouvelles données est trés compliquée. C'est pourquoi l'état de l'art est à l'heure actuelle encore trés restreint et est surtout très dur à faire évoluer. Nous présenterons dans une première partie le contexte des accidents graves lors de réaction nucléaire, puis nous présenterons le problème que soulève ce stage et les solutions qui pourront être apportées pour résoudre ce problème.