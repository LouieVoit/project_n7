Le projet \textbf{On Set Reconstruction Tool} a pour objectifs principaux de permettre à l'utilisateur de recréer en 3D une scène qu'il a prit auparavant en photo. 
Notre programme devra être capable d'aller chercher dynamiquement les photos présentent dans l'appareil photo de l'utilisateur (par USB et WIFI) et de relancer
une reconstruction. Nous avons, aprés une analyse approfondie des besoins des clients et discussions avec ceux ci, isolé deux points trés importants du projet.
Nous devorns modifier une librairie (OpenMVG) déjà 
existante ainsi que créer une intefarce graphique gérant dynamiquement les photos de l'utilisateur.\\
\begin{itemize}
\item La partie concernant la modification de openMVG a plusieurs enjeux : elle permettra de faire évoluer la bibliothèque déjà existante et en plus 
nous permettra de l'utiliser dans le cadre du projet. La bibliothèque ne permet actuellement que de faire des reconstructions à partir d'un ensemble 
d'images qui est fixé lors du lancement de l'algorithme. Notre objectif est de faire évoluer openMVG de telle sorte que l'on puisse
ajouter dynamiquement des photos même lorsque openMVG est lancé avec un ensemble de photos initial. Ceci implique bien évidemment
une compréhension assez profonde des algorithmes présents dans openMVG, mais aussi du code en C++ et donc du langage C++. Ceci a été,
lors de la lecture du sujet, l'un des atouts de ce projet : le C++ n'est malheureusement pas un langage enseigné à l'ENSEEIHT alors que 
celui ci est grandement utilisé dans l'industrie de nos jours. Mais ceci, la compréhension du code, est aussi un problème car
celui ci n'est pas très bien documenté et est assez "compliqué", du moins avec le peu de temps que l'on a pour le projet long, et demande
quelques notions de mathématiques orientées image et multimédia. \\
\item La partie concernant l'interface graphique soulève quant à elle beaucoup de questions : dans le document despriptif du problème, 
le langage utilisé pour coder cette interface est Qt, mais après quelques recherches et quelques discussions avec le client, il s'avère qu'il
y a beaucoup de variantes possibles (QtQuick, QML ...), et ce sont des langages d'IHM que nous n'avons pas du tout aboder à l'école. Ceci est
bien sur un atout car cela permettra d'élargir nos connaissance, mais en temps limité, cela représente aussi un vrai défi. \\
\end{itemize}

Après de nombreuses incompréhensions avec le client, le code rédigé sera principalement en python3 : langage qui n'est pas non plus
enseigné à l'école. Ainsi, ce projet nous permet réellement beaucoup d'apprendre mais s'annonce difficile au vu de ce que l'on à a faire
et du temps trés limité. Par rapport aux besoins des clients, nous avons eu dans un premier temps l'idée de ne pas modifier 
openMVG et de se limiter à une reconstruction dont l'ensemble d'images initiales est fixe (on ne peut pas rajouter de photo dynamiquement). 
On modifiera dans un second temps cette bibliothèque.