Nous avons bien avant le début du projet réfléchi à l'assignation des rôles dans notre équipe car c'est l'un 
des points importants de la gestion de projet et un point qui pourra nous permettre de mener à bien notre projet. 
De plus, dés la réunion avec les clients, nous avons vite compris que les clients étaient assez exigant sur 
la plannification, sur la qualité de notre travail (qualité du code, respect des règles de programmation) et 
surtout sur la communication entre le groupe et les eux-mêmes. Ainsi nous avons tout de suite décidé d'avoir
une chef de projet, un responsable qualité et un responsable intégration qui devront faire respecter des règles
à définir. 
Notre équipe est composé de 5 personnes dont moi : \\
\begin{itemize}
\item Arthur Manoha
\item Matthias Benkort
\item Nicolas Gaborit
\item Matthieu Pizemberg
\item Louis Viot\\
\end{itemize}

	Arthur Manoha a été désigné comme chef de projet car c'est celui qui a le plus d'expérience dans le travail en groupe
et dans sa gestion : il a en effet déjà effectué un stage en entreprise durant lequel il a travaillé dans un équipe de 
développement. Il a de plus déjà effectué un stage de recherche à l'école durant lequel il a aussi travaillé en équipe 
avec des professeurs expérimentés de l'école. Il nous a donc semblé comme étant le candidat idéal pour ce poste. 

	Oûtre le chef de projet, il nous a semblé utile d'élire un responsable qualité et un responsable intégration. 
	
	Le responsable
qualité a un rôle important dans notre groupe : en effet nous avons défini beaucoup de règles concernant la façon dont nous 
allons programmer, coder (indentation, commentaires ...), règles que nous avons bien sur partagé avec nos clients, qui les ont accepté. 
Le responsable qualité doit donc faire respecter ces règles. Matthias Benkort ayant le plus d'expérience dans la programmation,
nous l'avons choisi dans ce rôle.

	Le responsable intégration doit faire en sorte que chaque partie écrite par les membres du groupe soit bien intégrée
	dans le projet. Aussi, nous avons, avec accord avec les clients, d'utiliser un gestionnaire de version décentralisé (GIT) pour 
	pour notre projet. Celui ci, bien que trés pratique, n'est, une fois de plus, pas réellement enseigné à l'école. Le responsable 
intégration devra donc rédiger quelques papiers sur ce gestionnaire et devrai effectuer une remise à niveau pour certain membre 
du groupe. Matthieu Pizemberg ayant déjà une grande expérience avec GIT, nous l'avons choisi pour être ce responsable qualité. 
		
		
		Nicolas Gaborit et moi même avons pour rôle généralement de gérer les réunions clients/industriels et surtout de rédiger 
	les comtpe-rendus de ces réunions. Nous gérons aussi les autres fichiers, tels que le planning des actions par exemple. Cette organisation dans le groupe nous semble optimale et pour l'instant fonctionne
	parfaitement. 